\chapter{Conclusion}
\label{ch:conclusion}
The key task of GRAMOC is to process and visualise sensor data. The visualisation happens in real-time, which means it is well suited for monitoring solutions. An important feature of GRAMOC is the prediction of certain parameters, based on the input sensor data. With this predicted values it is possible to use GRAMOC as a quality inspection tool or as a tool to verify the status of certain events.

\section{Applications of GRAMOC}

\subsection{Steel Belt Quality Inspection}
The initial and main use case of the GRAMOC system is to verify the quality of steel belts during production. With the gathered data from highly sensitive MEMS gradient magnetometers, GRAMOC is able to predict the quality of steel belts. If the quality of the steel belts can be predicted in real-time, it is no longer necessary to halt production periodically to inspect a part of the steel belt to determine the quality. Without the need to halt the production it is way more efficient than before.

\subsection{Transport Driver Verification}
Another possible use case of GRAMOC could be the verification of transport drivers. For companies that employ drivers to transport any kind of cargo, it could be of interest to know how the drivers are driving. Another concern for such companies is to know who is driving their cars. For example, if a taxi driver is sick but does not want to loose his income for the day, he could send a family member to do his work. To prevent such occurrences, GRAMOC could be used to determine the driving profile of a driver and report the current driver if the driving style does not match the given driving profile.

\subsection{Sensor Monitoring}
A very generic application of GRAMOC is the monitoring of multiple sensors at once. GRAMOC could be used to create a dashboard for a broad variety of different sensors. This system would best fit into an environment where a lot of sensors need to be monitored. Such environments could be production lines or power plants. Especially in the case of a power plant it is critical to know if every machine is working properly. Another benefit that GRAMOC would add is a uniform design and user experience. If every sensor brings it own monitoring system, it could be complicated for employees to operate every system properly. GRAMOC could replace all of these systems and provide a uniform experience, which would simplify the tasks of employees.

\section{Outlook}
Many systems to gather sensor data that are used in the industry today feature different approaches for different sensors. The difference in how to control a specific sensor within one system can lead to many difficulties. To resolve that problem and to remove these difficulties, GRAMOC could be extended and implemented. Because with GRAMOC the controls of each sensor could be centralised and therefore it would be much easier to operate the whole system. But to use GRAMOC in such a large scale, it is necessary to implement additional features like:

\begin{itemize}
    \item Additional Types of Sensors
    \item Additional Visualisation Methods
    \item Additional Features
\end{itemize}

\subsubsection{Additional Types of Sensors}
In the future GRAMOC should support a broad variety of sensor types. GRAMOC features abstraction layers, through which it is relatively easy to implement support for new sensor types. With the addition of new sensor types the field of application grows and GRAMOC could become a common solution to visualise sensor data.

\subsubsection{Additional Visualisation Methods}
If more sensors are included in the system, there will be at some point the need for more features or ways how to visualise data. Data from certain sensors can not be properly displayed in two dimensions. Therefore it will be necessary at some point to enable three or more dimensional data representation. This would be one of the most prominent features that need to be added in the future. Other important features than adding support for more dimensions, will be support for different representations. Because certain sensor data needs to be depicted in more than one representation to create relevant results.

\subsubsection{Additional Features}
Other features could be more advanced statistical analyses, or intuitive ways to interact or convert the gathered data. The implemented multiple linear regression is only a first step to fully automated statistical analyses, that could be realised within GRAMOC. With the addition of more sensors, data could be predicted based on multiple sources which could lead to greater accuracy. Also if users, most likely scientists, need to inspect the data or display different representations of the data, it is necessary to implement functions to transform the data.
