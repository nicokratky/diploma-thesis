\chapter{Kurzfassung}
\begin{german}

Diese Diplomarbeit stellt GRAMOC vor. GRAMOC ist ein System das es ermöglicht Stahlbänder, ohne die Produktion stoppen zu müssen, effektiv zu charakterisieren. Dies wird durch die Erfindung eines hochsensitiven MEMS Gradienten Magnetometer ermöglicht.

Diese Prozedur steigert nicht nur die Produktivität, sondern ist auch material- und kosteneffizient.

Die ersten Versuche diese Problemstellung zu lösen wurden mit einem TCP-basiertem Server und einer nativen Android Applikation unternommen. Schnell stellte sich heraus, dass dies nicht die geforderten Echtzeitkriterien erfüllen kann. Diese Erfahrung resultierte in einem Neustart des Projektes mit geänderten Anforderungen. Die Android App wurde durch eine Webanwendung mit responsiven Design ersetzt. Dies erweitert nicht nur die Anzahl der unterstützen Endgeräte, sondern bringt auch Vorteile in punkto Drittanbieter Visualisierungsbibliotheken.

GRAMOC besteht letztendlich aus zwei Kommandozeilenprogrammen und einer Web Applikation. Das erste Kommandozeilenprogramm ist der Server, der die vom Sensor empfangenen Daten vorverarbeitet. Dieses Programm führt auch eine dynamische Datenanalyse durch. Das zweite Kommandozeilenprogramm kümmert sich um die Datenspeicherung, da alle Sensordaten in HDF5 Dateien gespeichert werden müssen, um eine nachträgliche Inspektion der Daten zu ermöglichen. Beide Programme laufen auf einem Raspberry Pi 3 Model B. Das Client Programm ist eine Web Anwendung die dazu dient, Sensordaten visuell darzustellen. Auch wird ein Formular zur Verfügung gestellt, um historische Sensordaten aus den HDF5 Datein abfragen zu können. Die kabellose Übertragung von Daten erfolgt über ein Wireless LAN Netzwerk, unterstützt durch ein eigens entwickeltes UDP-basiertes Netzwerkprotokoll.

Die empfangenen Sensordaten werden mittels multipler Linear Regression analysiert. Dies ermöglicht es, mechanische Parameter des Stahlbandes von den magnetischen Daten abzuleiten.

\end{german}
