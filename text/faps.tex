\chapter{Filtering and Preprocessing System}
\label{ch:faps}

\author{Nico Kratky}
%
FaPS, which stands for Filtering and Preprocessing System, is the main application of the GRAMOC backend. It is an application that reads digital sensor output, preprocesses it and forwards it to another process to distribute it to the final clients. It also saves the readout to HDF5 files \autocite{HDF5}.

The preprocessing consists of regression analysis (see \vref{ch:dataanalysis} for further details.)

\section{Command Line Interface}

As FaPS is a command line program, arguments that are passed to it have to be parsed. This is done by Boosts program\_options \autocite{BoostProgramOptions}. This module allows easy parsing and exception handling.

The arguments that FaPS can handle are depicted in table \vref{tab:faps_arguments}.

\begin{table}[h]
    \centering
    \begin{tabular}{| l | l | l | p{5cm} |}
    \hline
    \textbf{Flag} & \textbf{Argument} & \textbf{Default value} & \textbf{Description} \\ \hline
    -h, --help & & & Outputs the usage information \\ \hline
    --version & & & Prints version information \\ \hline
    -l, --loglevel & \textit{loglevel} & info & Information granularity during runtime \\ \hline
    --ip & \textit{ip\_adress} & 127.0.0.1 & IP address to which FaPS will connect to read sensor data \\ \hline
    --in-port & \textit{port number} & 9760 & Port to which FaPS will connect to read sensor data \\ \hline
    --out-port & \textit{port number} & 1337 & Port for the started UDP Server \\ \hline
    -s, --save & \textit{path to faps-save} & & Path to the faps-save executable \\ \hline
    -f, --filename & \textit{filename} & & Filename to which the data will be saved \\ \hline
    -p, --predict & & & Enter predict mode \\ \hline
    -c, --config & \textit{filename} & & Regression config file (only needed when -p is supplied)  \\ \hline
    -d, --driver & \textit{Driver ID} & 1 & Only needed when -p is not supplied \\ \hline
    \end{tabular}
    \caption{Flags that can be set, which arguments they take, their default values and what they do or change}
    \label{tab:faps_arguments}
\end{table}

\section{Data Storage}

To be able to offer a possibility for further data inspection all received data is saved to HDF5 files. This is done when both -s and -f parameters are supplied. If either of those is supplied solely, a error is printed and FaPS exits. The data is saved by a seperate process that acts as a client to FaPS. This seperate program is described in chapter \vref{ch:faps-save}.

\section{Data Processing}

% data is received as 1200 bytes, converted to 600 long array of shorts

As the sensor data is received in a raw binary format, it has to be processed in order to use it in a simple way.

One datagram that is received from the sensor is 1200 bytes long and consists of 600 shorts. This data is in network byte order. By convention, network byte order is always big-endian, which means that the most significant bit is placed first \autocite{IBMByteOrder}. The conversion of a two bytes in network byte order to a short is shown in code listing \vref{code:short_conversion}.

\begin{minipage}{\textwidth}
\begin{lstlisting}[caption={Conversion of two bytes to a short}, label={code:short_conversion}, captionpos=b, language=C++]
short s = (((short) bytes[i]) << 8) | bytes[i+1];
\end{lstlisting}
\end{minipage}

\section{Data Serialization}

In order to send data so that the other end can interpret the message it has to be packed into a common format. To do this JSON is used \autocite{rfc8259}. It is a language- and platform-independent data serialization format
originally specified by Douglas Crockford in the early 2000s. JSON is uses text in a human-readable form stores data in key-value pairs. Originally it was derived from JavaScript but as of today many programming languages
include methods and functions to en- and decode JSON. This integration in JavaScript is the biggest advantage for GRAMOC and ultimately led to the decision to use it in this project.

\section{Data Distribution}

Once preprocessing and serialization is finished, the received sensor data has to be distributed to connected clients. This is a typical use case for inter-process communication.

\subsection{Unix Domain Sockets}

Unix domain sockets are a way of communicating between a client and a server that are on the same host. There are two types of sockets available: stream sockets (compareable to TCP) and datagram sockets (compareable to
UDP). Although this solutions seams ideal for this, it was quickly decided against it. The reason for this decisions is that NodeJS does not support datagram unix domain sockets anymore and real time application usually
make use of datagrams. Therefore regular UDP was used.

\subsection{Solution}

This problem led to the decision to use a custom UDP protocol for data transfer. This custom protocol is discussed in chapter \ref{ch:faps-networking}.

\section{Storing data between starts}

In order to continue the regression analysis after FaPS has been stopped, a way of storing the regression info had to be developed. This was achieved by writing the data of the MLR class to a file. This process includes serialising the $ M $ matrix and $ V $ vector to arrays. The matrix is serialised by creating a two dimensional array where the inner arrays represent the matrix rows. So a $ \begin{bmatrix} 1 & 2 & 3 \\ 4 & 5 & 6 \\ 7 & 8 & 9 \end{bmatrix} $ matrix would become \lstinline{[[1, 2, 3], [4, 5, 6], [7, 8, 9]]}.
