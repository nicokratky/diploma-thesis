\chapter{Abstract}

This diploma thesis introduces GRAMOC, a system that can help to effectively characterise steel belts without halting the production lines. This is made possible by using a recently invented MEMS gradient magnetometer to measure the magnetic field of produced steel belts.

This new procedure not only increases productivity, but is also a lot more material and cost efficient.

The first approaches to solving this problem were made using a TCP-based server and a native Android application. It quickly turned out that this solution would not satisfy the requested real-time criteria. This experience resulted in a project restart with changes requirements. The Android application was replaced with a webapplication with responsive design. This does not only extend the amount of supported devices, but also has some advantages with regards to third-party plotting libraries.

GRAMOC ultimately consists of two command line programs and one web application. One command line program is the server that processes the data that is received from the sensor. This program also performs dynamic data analysis using this data. The second command line program handles data storage, as all sensor data has to be saved to HDF5 files to allow further inspection. These two programs run on a Raspberry Pi 3 Model B. The client program is a web application that visualises the received sensor data. It also provides a form to the users to request historical sensor data from the HDF5 files. The data is transmitted wirelessly via a Wireless LAN network supported by a specially developed UDP-based networking protocol.

The received data is analysed using multiple linear regression. This analysis method allows to predict mechanical parameters of the steel belt from magnetic data that is received from the sensor.

