\chapter{Measurement Results}
\label{ch:MeasureRes}

\author{Nico Leidenfrost}
%
In order to test GRAMOC, a test scenario had to be created. As an abstraction to the magnetometer, an accelerometer was used. The mathematical model to process the accelerometer sensor data is the same as the one form the magnetometer data. To get example sensor readouts from the accelerometer, the sensor was mounted inside a car. The sensor was mounted on the arm rest as depicted in figure \vref{fig:CarMount}.

\begin{figure}[H]
    \centering
    \includegraphics[width=10cm,keepaspectratio]{car_mount}
    \caption{Car Mount of the Raspberry Pi accelerometer}
    \label{fig:CarMount}
\end{figure}

\section{Test Scenarios}
To demonstrate the real-time plotting capabilities, a few scenarios were chosen in which it is easy to understand the recorded data. These scenarios are:

\begin{itemize}
    \item Shifting gears
    \item Driving in a roundabout
    \item Emergency braking
    \item Oversteering
\end{itemize}

As the used sensor is an accelerometer, the measured sensor data represent g-forces. The x axis represents the forward acceleration of the car, the y axis represents the lateral acceleration and the z axis represents the vertical acceleration. In the way the sensor was positioned, positive values on the x axis represent forward acceleration and negative values represent backward acceleration. Since the y axis displays the lateral acceleration, positive values represent a right turn and negative values represent left turns.

\subsection{Shifting Gears}
The first demonstration scenario is shifting gears while driving. Every time a driver shifts to another gear, the acceleration is interrupted. This short interruption is documented by the red line, which represents the forward acceleration, in figure \vref{fig:ShiftGears}. It can be observed that every time the driver shifts to another gear the line drops back to zero. The example in figure \vref{fig:ShiftGears} shows that the driver shifts 3 times from  first to fourth gear.

\begin{figure}[H]
    \centering
    \includegraphics[width=15cm,keepaspectratio]{shift}
    \caption{Measured sensor data when shifting gears}
    \label{fig:ShiftGears}
\end{figure}

\subsection{Driving in a Roundabout}
In the second scenario, a driver is driving through a roundabout. This scenario was chosen to be used as an example, because it is a situation where the lateral acceleration is roughly the same all the time. As shown in figure \vref{fig:Roundabout}, the forward acceleration stays around zero the whole time and the lateral acceleration is located about -0.5g. The reason why the acceleration remains stable the whole time is simply because when driving through a roundabout, a driver should drive with constant speed and turn.

\begin{figure}[H]
    \centering
    \includegraphics[width=15cm,keepaspectratio]{roundabout}
    \caption{Measured sensor data when driving in a roundabout}
    \label{fig:Roundabout}
\end{figure}

\subsection{Emergency Breaking}
Another scenario where acceleration can be easily shown is the case of emergency braking. When a driver needs to stop his car immediately because there are for example people in front of the car, there needs to be a massive amount of negative acceleration, depending on the current velocity of the car. The acceleration force of such a maneuver is depicted in figure \vref{fig:EmBrake}. The example below shows a driver who is driving at a steady speed of 30 km/h at first and then stops the car immediately. While braking the measured g-force reached a maximum of -1.1g.

\todo{Redo test, -1.1g sind mmn zu wenig}

\begin{figure}[H]
    \centering
    \includegraphics[width=15cm,keepaspectratio]{brake}
    \caption{Measured sensor data when applying an emergency brake}
    \label{fig:EmBrake}
\end{figure}

\subsection{Oversteering}

The last scenario was chosen to show how a car behaves when it is temporarily out of control. This could happend during oversteer caused by a wet street. As depicted in figure \vref{fig:Oversteer} the forward acceleration is erratic, but above zero the whole time during the oversteer phase. The green line which represents the lateral acceleration fluctuates between 1.0g and -1.0g for a short duration, in the remaining time the line stays below zero, due to the fact that the maneuver was applied in a left turn.

\begin{figure}[H]
    \centering
    \includegraphics[width=15cm,keepaspectratio]{drift}
    \caption{Measured sensor data when oversteering}
    \label{fig:Oversteer}
\end{figure}

\section{Regression Results}

\todo{Ausbauen, sehr wichtiger Teil!}

% was genau wurde gemacht?
% wie verhaelt sich der test zu den stahlbaendern?
% wie wurde untersucht (zwei fahrer, einer langsam einer schnell)
% was wurde berechnet?
% variablen, x y Beschl, timestamp (warum z Beschl unwichtig?)

% Ergebnisse
% warum funktionierts auf 'normaler' strecke nicht so gut? (gleichbleibende geschw => 0 beschl (sind ja brave fahrer die sich an die stvo halten ;-)))
% wie gut funktionierts im kreisverkehr (mittelwert? stdev? ausreisser?)

An important feature of GRAMOC is to be able to predict data through multiple linear regression. An important parameter to determine if the model produces correct results is the standard deviation of the predicted values during the tests. Because if the predicted values are very different it indicates that the model does not have a good fit.

% To be able to determine whether a regression model is accurate or not the $ r^2 $ test needs to be done. Another important parameter is the standard deviation of the predicted values during the tests, because if the predicted values are very different it indicates that the model does not have a good fit either.

To test this in the scenario where an accelerometer is mounted inside a car, data was first collected to initially train the model and then the driver was predicted based on the driving style of the current driver. As an abstraction there were only two types of drivers, a slow one and a fast one. A fixed course was selected to conduct the tests. During the tests the relevant variables were the acceleration of the x axis, the acceleration of the y axis and the current timestamp. The acceleration of the z axis was irrelevant because this axis indicates the acceleration up and downwards. Both the x and y axis values were used to determine which driver was driving at the moment. The timestamp was needed to bring the sensor data in a relation to the conducted tests.

First the slow driver completed the course a couple times and then the fast driver did the same. To distinct the drivers they were assigned by a number, the fast one had number 1 and the slow one got number 2. As the training of the model was finished, one of the drivers was selected to complete the course another time and the system had to predict which driver is currently driving the car.

The results showed that on a course with straight roads, the prediction of the driver was nearly impossible, because the acceleration on a straight road was about the same for both drivers. The acceleration for both drivers was the same because both drove with constant velocity and this means no change of acceleration. Since the velocity was not measured during the tests, on straight roads it was impossible to predict the driver. If the tests are conducted while driving in a roundabout, the results showed that the two drivers could be predicted very well. The reason why the prediction works much better in a roundabout is, because then the driver is constantly turning and there is a constant acceleration, unlike on a straight road.

Since GRAMOC works with streaming data, only the average of the predicted values was significant. That is because, the single values represent the prediction for only the time point where the values were measured and not for the whole run. The standard deviation of the average predicted values was about 10 percent.

The mathematical model to predict the driver is the same model that is used to predict the quality parameters of steel belts. This is the reason why this abstraction was chosen to be tested. The only modification that needs to be done is the usage of another sensor and the adjustment of the independent variables of the regression model. Since the base regression model does not need to change it is only a very small amount of work to implement this system in a steel belt production line.

