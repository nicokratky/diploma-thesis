\chapter{Android}
\label{ch:Android}
Android is a mobile operating system developed by Google. It is based on the Linux kernel. Android's primary focus is on mobile handheld devices with a touchscreen. The most popular examples would be smartphones, tablets and everything in between, like phablets and stuff like that. Android is open source which means developers can modify the underlying operating system as they wish. Android programs are called ``apps'' which is the short version of application, these applications extend the basic functionality of an Android device.

\section{History of Android}
Android started as a startup Company under the name ``Android Inc.'' It was founded in Palo Alto, California in October 2003. It was meant to be an advanced operating system for digital cameras. In 2004 they changed their goals to expand their operating system to handheld devices that can compete against Symbian and other mobile operating systems which were state of the art at this time. In July 2005 Google bought the whole company along with it's key employees for a huge amount of money, for at least 50 Million U.S. dollar. They developed an prototype which had similarities with a BlackBerry phone, it had no touchscreen and a physical QWERTY keyboard. Due to the launch of the Apple iPhone in 2007, Google changed its Android specification documents to state that "Touchscreens will be supported", although "the Product was designed with the presence of discrete physical buttons as an assumption, therefore a touchscreen cannot completely replace physical buttons". The first commercially available smartphone using Android as it's operating system was the HTC Dream announced in 2008. Since then Google launched numerous updates which improved the operating system bit by bit. They fixed bugs from prior releases and added new features along the way. Android major versions also have a naming scheme, they are all named after a dessert or a sugary treat. Each version starting with the next character in the alphabet starting with version 1.5 called ``Cupcake'', followed by 1.6 as ``Donut'' up to 7.0 as ``Nougat'' and the current version 8.0 as ``Oreo''. Google explained this naming scheme with the following sentence ``Since these devices make our lives so sweet, each Android version is named after a dessert''.

\section{Overview of Android Application Development}
 Applications are often abbreviated with ``apps''. These Android apps are written using the Android software development kit. There are a small selection of programming languages available that can be used to develop a native Android app.

\subsection{Java}
Java is the most popular language to develop an Android application. The majority of apps and libraries are written in Java. These apps are compiled to
But Java is not the only programming language used to create Android applications.
\subsection{C/C++}

\subsection{Go}

\subsection{Kotlin}
You may extend your apps with C or C++ code if you need to implement parts of your application in native code to get a performance boost. If you want to use this method you must work with the Android native development kit, short NDK and with the Java native interface, short JNI.  This is a very popular solution for gaming apps or applications that need a lot of computing power because if you write native code you can squeeze out that bit of extra power of every device that Java can't get you. Or you need the native code because of libraries that are already there and you don't want to rewrite them in Java. However there are more than these two programming languages supported. The Go language is also supported but you are limited to a smaller set of application programming interfaces, called API's. In May 2017, Google announced official support for the Kotlin Programming language.