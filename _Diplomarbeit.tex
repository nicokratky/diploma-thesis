\documentclass[english,oneside,color]{htldipl}
% Zulässige Class Options:
%   Hauptsprache: german (default), english
%   Doppelseitig: oneside (default), twoside
%   Syntax-Highlighting: color (default), black

% die folgende Zeile einkommentieren für Arial-Ähnliche Schriftart
%\renewcommand{\familydefault}{\sfdefault}

\graphicspath{{images/}}    % wo liegen die Bilder?

\usepackage{geometry}
\usepackage{hyperref}
\usepackage{float}

\geometry{
	a4paper,
	margin=3cm,
	top=1.5cm
}

%%%----------------------------------------------------------
\begin{document}
%%%----------------------------------------------------------

\title{GRAMOC - Gradienten Magnetometer Online Controller}

\abteilung{Informatik}
\schwerpunkt{}
\studienort{Wiener Neustadt}
\schule{HTBLuVA Wiener Neustadt}
\schullogo{htl.jpeg}
\abgabejahr{2017/18}

\betreuerA{Dr. Michael Stifter}
\betreuerB{}
\betreuerC{}
%\betreuerD{} leer lassen wenn nicht vorhanden

\schuelerA{Nico Kratky}
\evidenzA{5CHIF-13}
\subthemaA{Entwicklung Server, Netzwerkprotokoll}

\schuelerB{Nico Leidenfrost}
\evidenzB{5CHIF-14}
\subthemaB{Entwicklung App, Visualisierung}

\schuelerC{}
\evidenzC{}
\subthemaC{}

\schuelerD{}
\evidenzD{}
\subthemaD{}

\schuelerE{}
\evidenzE{}
\subthemaE{}

%\schuelerE{} leer lassen wenn nicht vorhanden
%\evidenzE{}
%\subthemaE{}

%%%----------------------------------------------------------
\frontmatter
\maketitle
\newgeometry{
	a4paper,
	margin=3cm
}
\listoftodo\clearpage
\tableofcontents
%%%----------------------------------------------------------

\chapter{Kurzfassung}
\begin{german}

Diese Diplomarbeit stellt GRAMOC vor. GRAMOC ist ein System Stahlbänder, ohne die Produktion stoppen zu müssen, effektiv zu charakterisieren. Dies wird durch die Erfindung eines hochsensitiven MEMS Gradienten Magnetometer ermöglicht.

Diese Prozedur steigert nicht nur die Produktivität, sondern ist auch material- und kosteneffizient.

GRAMOC besteht aus zwei Kommandozeilenprogrammen und einer Web Applikation. Das erste Kommandozeilenprogramm ist der Server der die vom Sensor empfangenen Daten vorverarbeitet. Dieses Programm führt auch eine dynamische Datenanalyse durch. Das zweite Kommandozeilenprogramm kümmert sich um die Datenspeicherung, da alle Sensordaten in HDF5 Dateien gespeichert werden müssen, um eine nachträgliche Inspektion der Daten zu ermöglichen. Beide Programme laufen auf einem Raspberry Pi 3 Model B. Das Client Programm ist eine Web Anwendung die dazu dient, Sensordaten visuell darzustellen. Auch wird ein Formular zur Verfügung gestellt um historische Sensordaten aus den HDF5 Datein abfragen zu können.

Die empfangenen Sensordaten werden mittels multipler Linear Regression analysiert. Dies ermöglicht es, mechanische Parameter des Stahlbandes von den magnetischen Daten abzuleiten.

\end{german}

\chapter{Abstract}

This diploma thesis introduces GRAMOC, a system that can help to effectively characterise steel belts without halting the production lines. This is made possible by using a recently invented MEMS gradient magnetometer to measure the magnetic field of produced steel belts.

This new procedure not only increases productivity, but is also a lot more material and cost efficient.

The first approaches to solving this problem were made using a TCP-based server and a native Android application. It quickly turned out that this solution would not satisfy the requested real-time criteria. This experience resulted in a project restart with changes requirements. The Android application was replaced with a webapplication with responsive design. This does not only extend the amount of supported devices, but also has some advantages with regards to third-party plotting libraries.

GRAMOC ultimately consists of two command line programs and one web application. One command line program is the server that processes the data that is received from the sensor. This program also performs dynamic data analysis using this data. The second command line program handles data storage, as all sensor data has to be saved to HDF5 files to allow further inspection. These two programs run on a Raspberry Pi 3 Model B. The client program is a web application that visualises the received sensor data. It also provides a form to the users to request historical sensor data from the HDF5 files. The data is transmitted wirelessly via a Wireless LAN network supported by a specially developed UDP-based networking protocol.

The received data is analysed using multiple linear regression. This analysis method allows to predict mechanical parameters of the steel belt from magnetic data that is received from the sensor.



% \chapter[Logo]{}
% \vspace*{\fill}
% \begin{center}
% 	\makebox[\textwidth]{\includegraphics[width=0.6\paperwidth]{gramoc-icon}}
% \end{center}
% \vspace*{\fill}
% \clearpage

%%%----------------------------------------------------------
\mainmatter           %Hauptteil (ab hier arab. Seitenzahlen)
%%%----------------------------------------------------------

\chapter{Introduction}
\label{ch:Introduction}

% what are we trying to solve / what is the problem?
% why is it important?

% minimize faulty material
% change contact pressure of steel belt rolls if something's going wrong
% cost reduction
% automatization (IoT, Industry 4.0)

% perform quality checks with sensors
% mechanical parameters can be calculated from magnetic field data
% highly sensitive MEMS gradient magnetometer
% try to predict faulty material in real time

Industry is ever-changing. Especially people working in the information technology branch know that, since these are the people that have to upgrade the current systems using latest technology. The latest industry-changing milestone was the rise of the so-called Industry 4.0, which combines regular mechanical processes with modern information and communication technology.

Industry 4.0 is a term that was coined by the German government \autocite{Industrie4.0Paper}. It describes the fourth industrial revolution. The first industrial revolution took place around 1800 with the rise of steam and water-powered machines. One century later electricity heralded the start of the second industrial revolution, production lines being one of the biggest milestones. Also division of labour was first practiced. The third industrial revolution occurred with the invention of computers, robots and computer automation. The fourth and final one basically just refines the third revolution. This revolution includes the term "cyber-physical systems", which are systems that are controlled by computers, algorithms and sensors. This also means that there has to be some kind of communication between these systems which happens mostly over the internet. Figure \vref{fig:industry40} depicts this sequence of revolutions.

\todo{Cite Industry 4.0 history (https://www.lmis.de/im-wandel-der-zeit-von-industrie-1-0-bis-4-0/)}

\begin{figure}[H]
    \centering
    \includegraphics[width=11cm,keepaspectratio]{industry40}
    \caption[The four industrial revolutions that took place over the last centuries]{The four industrial revolutions that took place over the last centuries\footnotemark}
    \label{fig:industry40}
\end{figure}

\footnotetext{\cite{img:industry4.0}}

This drastic change means that many companies have to adapt. These adaptions had to be made to keep up with the competing companies that already have these technologies.

Steel belt production companies are no exception. With the invention of a gradient magnetometer that can effectively characterize steel belts the foundation for this diploma thesis were laid.

\todo{cite patent? lol}

\section{Task}

The task of this diploma thesis is to develop a system to read sensor data, process it and visualize the results. Sensor data is continuously read from a highly sensitive MEMS gradient magnetometer. This data is structured as raw binary data and has to be processed by the system. The processed data will also undergo statistical analysis to predict parameters on the basis of this data. After this processing step the data has to be sent wirelessly to a mobile app. This mobile app acts as the client-side of the system. The app visualizes the sensor data and its predicted parameters. The app should also offer a way of browsing through historical data that was saved prior.

\todo{refactor paragraph, tenses, should}

\subsection{Requirements of GRAMOC}
To achieve an efficient visualization of the extensive amounts of sensor data transmitted over network, GRAMOC needs to fulfill a certain list of requirements.

Server side requirements:

\begin{itemize}
    \item test
\end{itemize}


Client side requirements:

\begin{itemize}
    \item test
\end{itemize}

\section{Current Solutions}

% continuous quality inspection of steel belts
% current solution:
%   produce a certain amount
%   take sample
%   use sample to perform quality checks (stretch tests, ...)
% waste of material / low yield
% time consuming
% not automatable

% real time plotting
% a lot of solutions exist
% little solutions transfer data over network, many assume that sensor is attached to computer
% also no big data volume

\subsection{Existing Solution for Steel Belt Quality Inspection}

Currently there are no solutions for dynamic steel belt characterization. All these measurements have to be made manually.

The current procedure to inspect the quality of a steel belt is as follows: The first thing that has to be done is to produce a roll of steel belt. To get the quality level of this product, a sample has to be taken from it. There are two samples taken from each steel belt roll, one from the start and one from the end. These two samples can now undergo quality inspection procedures. The results from these tests can be used to assess the produced steel belt. According to these tests, the parameters of the production machines can be adjusted.

This procedure has a few major disadvantages. Firstly, if the product does not pass the quality tests, the whole steel belt has to be discarded. Time and personnel are also two big disadvantages. These quality tests are not only time consuming but they also require special trained staff for conducting these inspections.

\subsection{Existing Solutions for Handling Sensor Data}

Currently there are a lot of solutions available that can plot sensor data. The majority of these is even free. The one constraint that most of these solutions share is that the sensor has to be directly connected to the computer. As the sensor that is used for this project sends its data over the network, almost all solutions are considered irrelevant. Also some custom features are wanted that these programs do not offer. For example visualizing historical data.

\subsection{Existing Solutions for Plotting Real Time Data}

As already mentioned there are a lot of solutions out there that can be used to plot sensor data. But the one thing that these solutions mostly can not provide is real time plotting. Static plots can be achieved in many different ways, with big amounts of data or just small amounts, many plots combined or divided in separate plots and many more variations are within the bounds of possibility. A lot of these solutions promote themselves with ``dynamic data updates'' or ``streaming data''. That just means that the data can be changed at runtime and therefore some could say the data is displayed in real time. But real time can be defined very differently. As one would say real time applications can update their data once every second, others consider that the data must be updated within less than 20 milliseconds to achieve a high framerate. Most of the solutions available can handle the former definition of real-time but nearly none of them can provide enough performance for the latter. Another important point is the amount of data that one wants to depict, because most of the already existing programs that can handle real-time are just powerful enough to handle small amounts of data.

\section{Outline}

% OUTLINE
% short summary of thesis
% explain two phases (I & II)
%   first phase => experimental, learning by doing
%   second phase => 'we know what we were doing', better planned

This diploma thesis is structured into two big parts. These parts can be seen as two phases of implementation. Each phase is completely separate. The first phase is a more experimental one as both authors were unfamiliar with these types of projects, so some experience had to be made. At the end of this phase there was a big cut and the project was restarted from the beginning. The second phase discusses the different approaches and decisions that were made starting from this cut. The second phase was not only better planned, but the decisions that were made, were mostly made out of experience from the first phase.

\chapter{Project Management}
\label{ch:Project Management}

\section{Kanban}

\subsection{Description of Kanban}
Kanban is a technique to manage software development in a project. It utilizes the just-in-time (JIT) production system and easily reveals possible bottlenecks in the software development pipeline. Kanban provides information visual instead of textual, therefore the brain can extract information much faster. This has a reason, the human brain is able to processes visual information 60 000 times faster than text \cite{WhatIsKanban}. Kanban uses Sticky notes or some sort of cards on a board to display the information, by this method the pure text information is packed in a bigger picture that the brain can now process faster.

\subsection{How Kanban works}
Kanban has four core principles \cite{WhatIsKanban}.
\begin{enumerate}
    \item \textbf{Visualize Work}

    By visualizing the work and workflow in a project, it's easier to observe the flow of work through the whole Kanban system. Also problems like bottlenecks, blockers and queues can be spotted immediately.

    \item \textbf{Limit Work in Process}

    Kanban limits the amount of unfinished work in process, by doing so it reduces the time one item needs to travel through the Kanban system. That brings the advantage of less task switching and constantly reprioritizing.

    \item \textbf{Focus on Flow}

    By implementing work-in-progress (WIP) limits and team-driven policies the Kanban system can be optimized to improve the workflow. With tis methods the flow can be analyzed and it can indicate possible future problems.

    \item \textbf{Continuous Improvement}

    Once the Kanban system is established, teams can track their effectiveness by measuring flow, quality and throughput of work items they are assigned to. This can help a team a lot to improve and take their effectiveness to a higher level.
\end{enumerate}

\subsection{History of Kanban}
Kanban was developed by Toyota in the late 1940s, the idea of Kanban came up from an unlikely source, the supermarket. They realized that the workers there ordered their goods if the stock was going near zero instead of order more goods if they are available at a vendor. So they ordered their goods after a ``just-in-time'' system. Kanban is Japanese for ``sign'' or ``signboard'', so Toyota invented a system that allowed teams to communicate better and showed them what work is needed to be done and when it's needed \cite{WhatIsKanban}.

\section{Scrum}

\subsection{Description of Scrum}
Scrum is a process framework to control and manage complex software development projects, it reduces the complexity of the whole system and focuses on a product that meets the business needs. ``Scrum makes clear the relative efficacy of your product management and development practices so that you can improve.'' \cite{ScrumGuide}

\subsection{How Scrum works}
Scrum consists of a number of entities that include:
\begin{itemize}
    \item \textbf{The Scrum Values} originally weren't included in the official Scrum guide, but due to most people considered them as a part of Scrum they were added in July 2016. These values are Commitment, Courage, Focus, Openness and Respect.

    \item \textbf{The Scrum Roles} are three roles every Scrum team needs to consist of. There is the ``Product Owner'', who needs to manage the Product Backlog. He needs to make sure the items are clearly expressed and in an order so that the team can achieve the best possible result. Theres also the ``Scrum Master'', who makes sure Scrum is used and understood properly. He also tells people outside of the Scrum team how they should interact with the team to maximize the output of the team. All of the other members in the Team are part of the ``Development Team''. Such a team is self-organized, cross-functional and there is no particular ranking in the team which means everybody is on the same level. The Development Team and only the Development Team is responsible to complete tasks from the Product Backlog at the end of each Sprint.

    \item \textbf{The Scrum Events} are used to create a regularity and minimize the effort of meetings outside the Scrum meetings. These events all have a fixed timespan and can't be shortened or lengthened. The Events embedded in Scrum are:
    \begin{enumerate}
        \item \textbf{Sprint}

        Each Sprint has a predefined timespan up to one month, in this period of time the Development Team should create a releasable increment to the product. An increment is the sum of all items in the Product Backlog that are finished within the Sprint. During the Sprint, nobody should make changes that would compromise the goal of the Sprint, also the stated quality of the product does not decrease. The scope of a Sprint can be clarified and re-negotiated with the Product Owner during the Sprint, when a Sprint finishes the next Sprint starts immediately.

        \item \textbf{Sprint Planning}

        The Sprint Planning Event takes place before each Sprint, in this meeting items from the Product Backlog are moved into the Sprint Backlog. These items doesn't have to be finished in one Sprint as a Sprint is defined by its time limit and not by the items in the Sprint Backlog.

        \item \textbf{Daily Scrum}

        In the Daily Scrum meeting the Development Team creates a plan for the next 24 hours. To do so the attendees analyze the work that has been done so far and with this knowledge they determine what to do until the next meeting. This meeting has a duration of at most 15 minutes and takes place every day in the same place at the same time.

        \item \textbf{Sprint Review}

        This meeting is held at the end of each Sprint, its purpose is to analyze the outcome and what items in the Product Backlog were actually finished in this Sprint. Attendees of this meeting are the Scrum team and the stakeholders, both groups should collaborate on the next things that could be done.

        \item \textbf{Sprint Retrospective}

        The main focus in this particular meeting lays on the improvement that can be done during the next Sprint. The topics discussed in the Sprint Retrospective are ``What went well in the Sprint'', ``What could be improved'' and ``What will we commit to improve in the next Sprint''.
    \end{enumerate}

    \item \textbf{Scrum Artifacts} represent the work that is to do is already is done to make it easier to inspect and adapt the particular work units. Therefore there are three Artifacts, the first one is the ``Product Backlog'', here defined are requirements that might be needed in the product. The Product Backlog evolves with the product itself, there will be requirements added and removed based on the results of the Sprints finished. The second Artifact is the ``Sprint Backlog'', in here are items that are assigned to do in the corresponding Sprint plus a plan how to accomplish the Sprint goal.
\end{itemize}
\subsection{History of Scrum}

\section{Scrumban}

\subsection{Description of Scrumban}

\subsection{How Scrumban works}

\subsection{History of Scrumban}

\subsection{Application of Scrumban in this Project}

\part{Implementationphase 1}
\chapter{Networking}
\label{ch:networking}

\author{Nico Kratky}
%
As sensor data is received over a network connection and it should also be delivered to clients wirelessly, a common way of communication had to be developed. This development process resulted in GSDEP, GRAMOC's networking protocol. It is a TCP-based networking protocol that is used for sending large amounts of sensor data \autocite{rfc793}.

\section{Data Flow}
\label{sec:networking_data-flow}

Figure \vref{fig:handshake} depicts the handshake performed by GSDEP that is based on TCP's three-way handshake. The client sends a synchronize (SYN) message to the server to let it know that it wants to connect. If the server can accept new Clients it returns an acknowledgment message (ACK). The client then also returns this acknowledgment message to inform the server that it is indeed connected. The connection now is established and data can be transmitted.

\begin{figure}[h]
    \centering
    \includegraphics[width=8cm,keepaspectratio]{gsdep_handshake}
    \caption{TCP-like three way handshake performed on client connect}
    \label{fig:handshake}
\end{figure}

If a client wants to disconnect from the server it will send a disconnect message (FIN) to the server. Before it actually disconnects, it has to wait for the server to finish cleaning up and return the FIN packet. After the client has received this message, it can close the connection and shut down. This procedure is shown in figure \vref{fig:disconnect}.

\begin{figure}[h]
    \centering
    \includegraphics[width=8cm,keepaspectratio]{gsdep_disconnect}
    \caption{Two way handshake performed on client disconnect}
    \label{fig:disconnect}
\end{figure}

\section{Data Interchange Format}

Messages have to be brought to a common format to be understood by all communication partners. Therefore every message transmitted is prefixed with a header. This header includes additional information that is used by the receiving end to determine the size of the payload (see section \vref{sec:messageframing}), to differentiate between different kinds of messages (see section \vref{sec:channels}) and to rebuild the message data to its correct data type. The header consists of 8 bytes, 4 bytes to store the payload length, and 2 bytes each for data type and channel. A example packet is illustrated in figure \vref{fig:packet}.

\begin{figure}[h]
    \centering
    \includegraphics[width=8cm,keepaspectratio]{gsdep_packet}
    \caption{Structure and field sizes of a packed message sent with GSDEP}
    \label{fig:packet}
\end{figure}

\section{Commands}
\label{sec:networking_command}

Commands are special message that are used to prompt the other end to do something. These commands are used for two purposes. On the one hand they are used during the connection establishment and termination phases, and on the other hand they are used to request data or to stop data transmission. These commands are listed in table \vref{tab:commands}.

\begin{table}[h]
    \centering
    \begin{tabular}{| l | l | p{5cm} |}
    \hline
    \textbf{Command} & \textbf{Used by} & \textbf{Meaning} \\ \hline
    SYN & client & Tells the server that a new client is waiting for the connection procedure \\ \hline
    ACK & server \& client & Tells the other end that it acknowledges the previous command \\ \hline
    FIN & server \& client & Tells the other end that it will disconnect \\ \hline
    STD & client & Tells the server that a client requests data\\ \hline
    SPD & client & Tells the server that a client does not want any more data\\
    \hline
    \end{tabular}
    \caption{Commands sent by one of the connection partners and what they do}
    \label{tab:commands}
\end{table}

\section{Channels}
\label{sec:channels}

In the case of GRAMOC, where large amounts of data are received in short periods of time, it is crucial to differentiate between communication data and sensor data in split seconds. To accomplish this, 2 bytes are included in the message header. This field simply contains numbers that represent different channels (see table \vref{tab:channels}). This information can then be used by the client to tell apart these two types of data, without even analyzing the payload.

\begin{table}[h]
    \centering
    \begin{tabular}{| l | c |}
    \hline
    \textbf{Channel} & \textbf{Value} \\ \hline
    Communication & 1 \\ \hline
    Data & 2 \\
    \hline
    \end{tabular}
    \caption{Channels used to distinguish between message types}
    \label{tab:channels}
\end{table}

\section{Message framing}
\label{sec:messageframing}

A common mistake that many developers make is to assume that TCP operates with messages and that TCP can tell apart these messages \autocite{MessageFramingCleary} \autocite{MessageFramingSkotzko}. Sadly this is not true as TCP operates with continous streams of data. Therefore the differentiation of messages has to be done by the developers. This can be achieved in two ways. These procedures are taken from the blog posts \citetitle{MessageFramingCleary} and \citetitle{MessageFramingSkotzko}.

\subsection{Delimiters}

\subsubsection{Sending}

Using delimiters probably is the simplest solution. This can be done by sending a special character between each message. This character can either be a character that does not show up in actual messages (e.g. a Null character), or a character that is present in a message. If the second approach is used, every message has to be run through an escaping process which replaces these characters in the messages.

\subsubsection{Receiving}

Receiving delimited messages is relatively straightforward. The program knows that a message has been fully read when it encounters a delimiter character. This message then has to be passed to an unescaping function when a delimiter character is chosen that can exist in messages.

\subsection{Length Prefixing}

Another method of message framing is to prefix each message with its length. When doing so the format of this prefix has to be stated explicitly. In the case of GSDEP that is a ``4 byte unsigned integer''.

\subsubsection{Sending}

First, the message has to be encoded into its binary representation. To send this message, the length followed by the binary encoded message simply has to be sent.

\subsubsection{Receiving}

Receiving one message is done by first reading into a buffer with the length of the length prefix (in this instance the buffer would be 4 bytes long). Then the payload is read into a second buffer with the just read length. When this buffer is full, one message has been read.

\subsection{Security Concerns}

Whichever solution is chosen, each solution has to provide code regarding Denial of Service (DoS) attacks. Wether a very big message length or large amounts of data without a delimiter are received, both can result in \textit{Out of Memory Exceptions}.

\chapter{Server}
\label{ch:server}

\todo{what are servers?}

\section{History of Servers}

\todo{what were the first approaches to servers? how are they used?}

\section{Tasks}
The tasks of the server are to manage incoming connection and their requests. It should always be prepared to send sensor data to a client if it is requested.

\section{Hardware}
\subsection{Raspberry Pi 3 Model B}
The Raspberry Pi is a small single-board computer originally created to teach children how to program \cite{RasPi}. It was developed by the Raspberry Pi Foundation. There are numerous options for expanding the capabilities of the Raspberry Pi. One being the use of General-purpose input/output (GPIO) pins. These can be used by so-called HAT's (Hardware Attached on Top) or Shields (this term evolved from Arduino-land). These add-on boards are mounted to the Raspberry Pi by connecting the GPIO pins to the board and screwing them together. They mostly provide additional hardware that can be used to achieve the desired goal. Further advantages are its relatively small footprint and its low cost. Also, a wide variety of Linux distributions have been adapted to the hardware. Having these advantages was the decisive factor for choosing the Raspberry Pi 3 Model B.\\
The specifications of the Raspberry Pi 3 Model B are:

\begin{itemize}
	\item \textbf{SoC} Broadcom BCM2837
	\item \textbf{CPU} 4x ARM Cortex-A53, 1.2GHz
	\item \textbf{GPU} Broadcom VideoCore IV
	\item \textbf{RAM} 1GB LPDDR2 (900 MHz)
	\item \textbf{Networking} 10/100 Ethernet, 2.4GHz 802.11n wireless
	\item \textbf{Bluetooth} Bluetooth 4.1 Classic, Bluetooth LE
	\item \textbf{Storage} microSD
	\item \textbf{GPIO} 40-pin header, populated
	\item \textbf{Ports} HDMI, 3.5mm analogue audio-video jack, 4x USB 2.0, Ethernet, Camera Serial Interface (CSI), Display Serial Interface (DSI)
\end{itemize}

\begin{figure}[H]
	\centering
	\includegraphics[width=10cm,keepaspectratio]{raspberrypi3b}
	\caption{Raspberry Pi 3 Model B}
	\label{fig:raspberrypi3b}
\end{figure}

\subsection{Raspberry Pi SenseHAT}
During the first implementation phase a Raspberry Pi SenseHAT add-on board was used the get real sensor data \cite{SenseHAT}. This HAT was made especially for the Astro Pi mission, where student could create and code projects, which were then run on the International Space Station by astronaut Tim Peake \cite{AstroPiMission}. This board was chosen because it offers a wide variety of sensors and therefore offers many possibilities in terms of testing GRAMOC.

The Raspberry Pi Sense HAT includes following sensors and inputs/outputs:

\begin{itemize}
	\item ST LSM9DS1 Inertial measurement sensor
		\begin{itemize}
			\item 3D accelerometer
			\item 3D gyroscope
			\item 3D magnetometer
		\end{itemize}
	\item ST LPS25H barometric pressure and temperature sensor
	\item ST HTS221 relative humidity and temperature sensor
	\item Alps SKRHABE010 5-button mini-joystick
	\item 8x8 RGB LED matrix
\end{itemize}
\bigskip
The RGB LED matrix and the joystick are both driven by a Atmel ATTINY88 microcontroller-unit.\\
Because this project focuses on data from magnetic field sensors, the focus was put onto the 3D magnetometer of the SenseHAT. This sensor has a magnetic measurement range of $\pm$ 4/8/12/16 gauss \cite{InertialSensorsManual}.

\begin{figure}[H]
	\centering
	\includegraphics[width=8cm,keepaspectratio]{sensehat}
	\caption{Raspberry Pi SenseHAT}
	\label{fig:raspberrypi3b}
\end{figure}

\subsection{Magnetic field sensors}

\todo{what are magnetic field sensors, what do the measure, how do they measure}

\section{Implementation}

The server program of GRAMOC is written in Python. This allows for great compatibility as Python comes preinstalled on many systems.

\subsection{Programming Language}

Python is a simple yet powerful, modern programming language and supports both procedure-oriented as well as object-oriented programming. It was developed by Guido van Rossum at Centrum Wiskunde \& Informatica (CWI) in the Netherlands in 1989 and first released in 1991 \cite{HistoryOfPython}. It was meant to be a successor to the ABC programming language. Python is a high-level language and therefore includes features such as automatic memory management.

Python is currently available in version 3.6.2. Nevertheless version 2.7 is still available as the Python Software Foundation announced that it will be supportet until 2020, effectivly making it an Long-term support version. Despite that Python 3.6.2 was chosen for GRAMOC as the Foundation also encourages users to use the newest version if possible. Another reason for choosing the newer version is that GRAMOC does not have to be backwards compatible to any existing software.

\section{Program Flow}

As depicted in figure \ref{fig:server-program-flow}, the server starts accepting new connections right after is has started. It then performs the handshake that is required by GSDEP (further explained in \ref{sec:networking_data-flow}). If this handshake is performed without errors, the server starts listening for data from this now connected client on a separate thread. While this thread is running it receives one message and checks if it is a command (see \ref{sec:networking_command}). If it is, the message is interpreted and the appropriate function is executed. This thread is kept alive until the client disconnects or the server is shutdown by the user.

\begin{figure}[H]
	\centering
	\includegraphics[width=8cm,keepaspectratio]{server-task}
	\caption{Flowchart of server program showing the procedure}
	\label{fig:server-program-flow}
\end{figure}

\chapter{Android}
\label{ch:Android}
Android is a mobile operating system developed by Google, based on the Linux kernel. Android's primary focus is on mobile handheld devices with a touchscreen. The most popular examples would be smartphones, tablets and everything in between, like phablets. Android is open source which means developers can modify the underlying operating system as they wish. Android programs are called ``apps'' which is the short version of application, these applications extend the basic functionality of an Android device.

\section{History of Android}
Android started as a startup Company under the name ``Android Inc.'' It was founded in Palo Alto, California in October 2003. It was meant to be an advanced operating system for digital cameras. In 2004 they changed their goals to expand their operating system to handheld devices that can compete against Symbian and other mobile operating systems which were state of the art at this time. In July 2005 Google bought the whole company along with its key employees for a huge amount of money, 50 Million U.S. dollar at least, according to rumors. They developed a prototype which had similarities with a BlackBerry phone, it had no touchscreen and a physical QWERTY keyboard. Due to the launch of the Apple iPhone in 2007, Google changed its Android specification documents to state that "Touchscreens will be supported", although "the Product was designed with the presence of discrete physical buttons as an assumption, therefore a touchscreen cannot completely replace physical buttons". The first commercially available smartphone using Android as its operating system was the HTC Dream announced in 2008. Since then Google launched numerous updates which improved the operating system bit by bit. They fixed bugs from prior releases and added new features along the way. Android major versions also have a naming scheme, they are all named after a dessert or a sugary treat. Each version starting with the next character in the alphabet starting with version 1.5 called ``Cupcake'', followed by 1.6 as ``Donut'' up to 7.0 as ``Nougat'' and the current version 8.0 as ``Oreo''. Google explained this naming scheme with the following sentence ``Since these devices make our lives so sweet, each Android version is named after a dessert''.

\section{Design}
Material Design is Google's visual design language that was first introduced in 2014. The goal was to develop a single underlying system that allows for a unified experience across all kinds of devices. It tries to support visual elements with the characteristics of real materials, hence Material Design. These guidelines help the users to interact and quickly understand different kinds of User Interface (UI) elements by using familiar tactile attributes.

GRAMOC's Android app uses these design principles for the user interface as shown in figure \ref{fig:appscreenshots}

\begin{figure}[H]
	\centering
	\begin{tabular}{cc}
	\includegraphics[height=7cm,keepaspectratio]{app_connect}
	&
	\includegraphics[height=7cm,keepaspectratio]{app_navdrawer}
	\\
	\includegraphics[height=7cm,keepaspectratio]{app_sensor}
	&
	\includegraphics[height=7cm,keepaspectratio]{app_about}
	\end{tabular}
	\caption{Screenshots of App}
	\label{fig:appscreenshots}
\end{figure}

\section{Overview of Android Application Development}
 Applications are often abbreviated as ``apps''. These Android apps are written using the Android software development kit (SDK). There are a small selection of programming languages available that can be used to develop a native Android app.

\subsection{Java}
Java is the most popular language to develop an Android application. The majority of apps and libraries are written in Java. These apps are compiled to bytecode which then will be translated to native instruction by the Android Runtime (ART). ART is an application runtime environment that replaced its predecessor Dalvik, a process virtual machine developed to run Android applications. Java was chosen to be used in this project because of the broad variety of third-party libraries available.

\subsection{C/C++}
With C or C++ Code and the Android native development kit (NDK), a native library for Android, applications can get much better results in terms of performance. Because the C or C++ Code runs natively on the device it executes faster than the Java code run in the Android runtime environment. The only downside of this is that all of the C or C++ code needs to be handled through the Java native interface (JNI). This programming framework handles the interoperability of the Java code and the C/C++ Code.

\subsection{Go}
The Go programming language is an open source project developed by a team at Google and many contributers from the open source community \cite{GoProject}. This programming language is supported although there are limitations to the application programming interfaces (API), therefore it was not considered a reasonable option for GRAMOC.

\subsection{Kotlin}
In May 2017, Google announced official support for the Kotlin Programming language. Kotlin is a modern and powerful language and solved some issues addressed with Java (e.g. Null references). Kotlin is also interoperable with Java which means Kotlin can be used in already existing Java projects. Kotlin was considered to be used in this project but the fact that the official support was only recently introduced and therefore the list of available third-party libraries is not as comprehensive as in Java, led to the decision that Java was the programming language of choice.

\subsection{Runtime}
A runtime refers a system that converts code written in a high level language like Java into CPU readable byte code. In Java the code is compiled to Java byte code and then converted to native code with the Java Virtual Machine (JVM). In Android however the Java code is compiled to Java byte code, then compiled again to Dalvik byte code and then given to the runtime. Two different types of a runtime were introduced in Android.

	\subsubsection{Dalvik}
	At the beginning of Android the Dalvik Virtual Machine (DVM) was used to execute Android applications. The main difference between the DVM and the JVM is that the DVM is register-based and the JVM is stack-based. Stack-based machines must use certain instructions to load and manipulate data on the stack. A register-based machine need less but more complex instructions than a stack-based machine. Dalvik uses just-in-time (JIT) compilation, so each time the application is run, Dalvik dynamically translates chunks of the Dalvik byte code into native machine code. These code parts are translated and cached if needed, therefore not the whole application is translated and this results in a smaller memory footprint.

	\subsubsection{Android Runtime Environment}
	The Android Runtime Environment (ART) was introduced in Android version 4.4 and later replaced Dalvik completely in version 5.0. The main difference to Dalvik is the use of an Ahead-of-Time (AOT) compiler instead of a JIT compiler. With AOT compilation the whole Dalvik byte code is translated to native machine code during the installation process of the application. This method improves the execution speed of the application because the CPU does not need to compile the Dalvik byte code during runtime. Also the startup time and battery consumption could be decreased with AOT compilation. The only drawbacks are that the installation time and the required storage space are increased.

\section{Components}
In order to build this Android application following Android components were used:

\begin{itemize}
	\item Intent
	\item Toolbar
	\item Activity
	\item Service
	\item NavigationDrawer
	\item Threads
\end{itemize}

\subsection{Intent}
An intent is an abstract description of an operation to be performed \cite{AndroidIntent}. It handles the execution of a specific action that it takes along with data to operate on. It's most used when launching a new activity.

\subsection{Toolbar}
This component is a widget from the Android \emph{appcompat support library} and is usually used as an app bar, also known as the action bar. The most important functions of such a bar is to make space for identification of an application and to create an easy way to perform important actions, like searching or navigating.

\subsection{Activity}
``An activity is a single, focused thing that the user can do.'' \cite{AndroidActivity} An activity is the main entry point of an application, it takes care of creating a new window and loading all the User Interface (UI) elements. Activities are usually shown as a full-screen window, but they can also be used as a floating window or even be embedded inside of another activity by implementing an ActivityGroup. Inside the Android-system, activities are managed as a stack, this means when a new activity is started it will be placed on top of this stack and become the running activity. The other activities are placed below this one in the stack and therefore remain inactive. The lifecycle of such an activity can be described as shown in Figure \ref{fig:activitylifecycle}.

\begin{figure}[H]
	\centering
	\includegraphics[width=10cm,keepaspectratio]{android-activity-lifecycle}
	\caption{Flowchart showing the lifecycle of an Android-activity}
	\label{fig:activitylifecycle}
\end{figure}

\subsection{Service}
A service is used to perform long-running operations in the background that do not need a user interface like an activity. Once started, a service can persist even if the user switches to another application. If another component binds itself to the service, it enables the possibility of interprocess communication (IPC). A typical example for a service is to handle network connections through it. The lifecycle is described as depicted in Figure \ref{fig:servicelifecycle}.

\begin{figure}[H]
	\centering
	\includegraphics[width=10cm,keepaspectratio]{android-service-lifecycle}
	\caption{Flowchart showing the lifecycle of an Android-service}
	\label{fig:servicelifecycle}
\end{figure}

\subsection{NavigationDrawer}
To navigate to the different activities and views a navigation drawer was implemented. In other words this is a panel that swipes in from the left side of the screen to approximately 3 quarters of the screen width and consists of a header where general information is displayed and a body which is filled with different navigation items. A navigation drawer is part of the material design pattern and is therefore commonly used in applications which implement this kind of design.

\subsection{Threads}
When an Android application component starts and no other component of the same application is already started Android will start a new Linux process. On the other hand if an application component is started and there is already a process of this application running, Android will launch the component in the main-thread of this application unless it's explicitly started in a new thread within this process. There are two essential rules to follow when dealing with threads in Android:

\begin{enumerate}
	\item Do not block the UI thread
	\item Do not access the Android UI toolkit from outside the UI thread
\end{enumerate}

\noindent The reasons behind this two rules are quite simple. The point why the UI thread should never be blocked is simply because then no events could be dispatched, including events that update the UI itself. This means the application would appear to be stuck and thats really bad. Accessing the UI toolkit outside the UI thread is also a bad idea, since the UI toolkit is not thread-safe. If the access would not be protected by this rule a race condition could happen and therefore cause errors within the application. To avoid such errors there are a few different ways how to execute tasks asynchronously in Android:

	\subsubsection{Extended Threads}
	The first possible solution is to create a subclass of the Java \emph{Thread} class. If this method is chosen the \emph{run} method of the superclass needs to be overwritten. This method is recommended if the behavior of the thread needs to be modified or new functionalities needs to be implemented, otherwise implementing the Runnable interface would be more appropriate.

	\subsubsection{Runnable interface}
	Another way to accomplish asynchronous behavior is to create a class that implements the \emph{Runnable} interface. An Instance of this class can then be passed to the specific thread in which the tasks are executed. This method is preferred to use when running tasks without the need of modified thread behavior because a runnable class won't need to create a new thread every time it executes. Also a runnable class can be executed on different threads.

	\subsubsection{AsyncTasks}
	AsyncTasks are used to do work in the background and then update the UI accordingly. As AsyncTasks are defined to do blocking operations there can only be one running AsyncTask at a time. There are four steps that will be executed when performing an AsyncTask:

	\begin{enumerate}
		\item \textbf{onPreExecute}: executed on the UI thread before the task is executed.
		\item \textbf{doInBackground}: executed on the background thread, here the background work that needs to be done is executed.
		\item \textbf{onProgressUpdate}: executed on the UI thread every time when \emph{publishProgress} is called in the background thread.
		\item \textbf{onPostExecute}: executed on the UI thread when the background tasks are finished.
	\end{enumerate}

\subsection{Libraries}
The Android client was implemented using a small number of libraries:
\begin{itemize}
	\item \textbf{Android SDK}

	The standard libraries included in the Android platform itself \cite{AndroidSDK}.

	\item \textbf{GramocAlgorithm-client}

	The Java implementation of the GSDEP client developed along with this project \cite{GramocAlgorithm-client}.

	\item \textbf{MPAndroidChart}

	An easy to use but also powerful open source 2 dimensional chart library for Android \cite{MPAndroidChart}.

	\item \textbf{android-about-page}

	This library allows to simply create an about page for your Android application \cite{android-about-page}.
\end{itemize}

\section{Implementation}
The entry point of this Android application is called the \emph{MainActivity}. When this Activity starts a background service is also started which is basically a wrapper for the GSDEP client, therefore it handles all the networking related tasks within the app. The service will be bound to the active activity, so every time another activity is launched the service will be unbound by the current activity and newly bound by the starting one. The \emph{MainActivity's} goal is to give the user an easy way to connect to the server. Once the application successfully connects to the server, a new activity responsible for plotting the received sensor data will be launched, whether the 2 dimensional or the 3 dimensional plotting activity is launched depends on the selection made in the \emph{NavigationDrawer}, by default the 2 dimensional plotting activity will be launched. When the 2 dimensional plotting activity is launched the networking service will be bound and three \emph{LineCharts} contained within the library \emph{MPAndroidChart} will be created and properly set up. After these tasks are finished and the activity is ready to receive data, the server will be notified. Now each data set received will be added to the data buffer of the respective chart. If the buffer of a chart is full, the values at the end will be truncated until there is enough space to add the new values. The 3 dimensional plotting activity however was not implemented at all, since the Android application was discontinued because of problems that appeared during the development of the 2 dimensional activity (see \autoref{ch:Problems}).


\part{Lessons Learned}
\chapter{Problems}
\label{ch:Problems}

\author{Nico Leidenfrost}
%
After extensive testing as described in section \vref{sec:Tests}, it was decided that this approach will not lead to a successful project outcome. This decision was made while taking several factors into consideration.

\section{Android}
Android is a great platform to create simple and even complex application systems that does not rely on heavy performance. Since the key element in this project is the ability to display the sensor data in real-time, the Android development was discontinued due to the performance issues that come with it.

\section{Software limitations}
Android is designed to render the UI with 60 frames per second (fps), which results in redrawing frames every 16 milliseconds at best. The task of drawing frames will be executed by the main thread along with many other operations like system events, input events, application callbacks and so on. The system tries to update the screen every 16 milliseconds, if however other operations than the redrawing of the screen are pulled from the work queue when trying to update the screen, these frames will simply be dropped and users will experience lacks of smoothness while using the application. To be sure about how much milliseconds the rendering of one packet takes the time was stopped. The results showed that it took up to about 50 to 60 milliseconds to render one update. These measurements were the prime factor that led to the decision to discontinue the work on Android.

\section{Plotting Libraries}
There are a lot of freely available plotting and charting libraries to use in Android development. Unfortunately most of them do not meet the requirements to be used in this project. There are many good libraries to plot 2 dimensional charts like pie charts or bar charts, but there is a lack of libraries that can display scientific plots (e.g. surface plots). The libraries that would meet all the requirements however are not originally designed to be used in Android development and therefore work only in specific versions of Android or do not work at all because they rely on components that are not available in Android.

\section{Networking}
The first approach of transmitting sensor data to the plotting application was the GSDEP protocol (see chapter \vref{ch:networking}). The goal of this protocol was to send sensor data reliable to clients such as the GRAMOC Android application. As it was meant to send data reliable, the TCP/IP stack was used as underlying technology \autocite{rfc793} \autocite{rfc791}. This approach worked well with small amounts of data, but since GRAMOC requires to send an enormous amount of data this approach failed, because the data was stuck in a buffer and could not be displayed in real-time.

\section{Tests}
\label{sec:Tests}
The two main test factors that lead to discontinuing the Android development in conjunction with the GSDEP protocol are:

\begin{itemize}
    \item The time of one chart update
    \item The time to empty the data buffer
\end{itemize}

\subsection{Update Test}
This test was used to determine the redraw or update performance of the Android application. To measure the time that one full update took, the timestamps of the start and end of an update circle were taken. The start time was then subtracted form the end time and the result was the time of one update cycle. These durations were measured in milliseconds. Results showed that one update took 50 to 60 milliseconds, that corresponds to 16 to 20 frames per second (fps). The goal was to achieve a smooth experience with about 50 to 60 fps and thats far away from what the test results revealed.

\subsection{Buffer Test}
During the testing phase it occurred that the displayed data values did not match the given data input at the time. Also the application kept displaying new data after the sensor stopped transmitting. These anomalies indicated that the huge amounts of data could not be plotted in real-time and ended up in a buffer. To measure the time it took to fully empty the buffer of the not drawn data, the sensor was constantly sending data for about five minutes and then stopped. The time between the sensor stopped transmitting data and the last chart update was taken and evaluated. The results showed that the application stopped updating the chart roughly about two minutes after the sensor stopped sending new data. Based on this test the whole networking stack was rebuild on top of UDP to achieve the desired speed.

\chapter{Résumé}

\author{Nico Kratky}
%
After researching alternatives that still fit the purpose of GRAMOC, a meeting with the client was arranged to discuss these alternatives. This meeting resulted in new goals and expected results. This new project specification now includes a web application instead of the mobile Android application.

\section{Advantages}
The switch to developing a web application still offers a few advantages that were not existing while focusing on a Android application. This includes the flexibility that a web application can run on basically any device the end user wishes. As of today many devices support network connections and can run a web browser. Another advantage is that JavaScript offers a tremendous amount of third party libraries, especially plotting libraries. A few of these libraries even support scientific plotting, like for example VTK, the visualisation toolkit that is used by ParaView \autocite{VTK}.

\section{Disadvantages}
The change of specifications also brings some disadvantages with it. For example the whole networking stack has to be rewritten because raw TCP streams are not supported in web environments. They were replaced be the WebSocket technology \autocite{rfc6455}. Also a new third-party plotting library has to be chosen and read up on.


\part{Implementationphase 2}

%Literaturverzeichnis
\clearpage
\addcontentsline{toc}{chapter}{\bibname}

\bibliographystyle{bababbrv}
%options: babplain, babunsrt, bababbrv, babalpha
%         babplain-fl etc: first/lastname
%         babplain-lf etc: last/firstname
\bibliography{references}     %BibTeX-Datei literatur.bib


%%%----------------------------------------------------------

%%%Messbox zur Druckkontrolle
\include{messbox}

\end{document}
