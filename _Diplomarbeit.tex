\documentclass[english,oneside,color]{htldipl}
% Zulässige Class Options:
%   Hauptsprache: german (default), english
%   Doppelseitig: oneside (default), twoside
%   Syntax-Highlighting: color (default), black

% die folgende Zeile einkommentieren für Arial-Ähnliche Schriftart
%\renewcommand{\familydefault}{\sfdefault}

\graphicspath{{images/}}    % wo liegen die Bilder?
\usepackage[margin=3cm]{geometry} % alle Seitenränder auf 3 cm einstellen

\makeglossaries
%\include{glossary}
\loadglsentries{glossary}
%%%----------------------------------------------------------
\begin{document}
%%%----------------------------------------------------------

\title{GRAMOC - Gradienten Magnetometer Online Controller}
\abteilung{Informatik}
\schwerpunkt{}
\studienort{Wiener Neustadt}
\schule{HTBLuVA Wiener Neustadt}
\schullogo{htl.jpeg}
\abgabejahr{2017/18}
\betreuerA{Dr.\ Michael Stifter}
\betreuerB{}
\betreuerC{}
%\betreuerD{} leer lassen wenn nicht vorhanden
\schuelerA{Nico Kratky}
\evidenzA{5CHIF-13}
\subthemaA{}
\schuelerB{Nico LEIDENFROST}
\evidenzB{5CHIF-14}
\subthemaB{}
\schuelerC{}
\evidenzC{}
\subthemaC{}
\schuelerD{}
\evidenzD{}
\subthemaD{}
\schuelerE{}
\evidenzE{}
\subthemaE{}
%\schuelerE{} leer lassen wenn nicht vorhanden
%\evidenzE{}
%\subthemaE{}

%%%----------------------------------------------------------
\frontmatter
\maketitle
\tableofcontents
%%%----------------------------------------------------------

\include{vorwort}				%ggfs. weglassen
\include{kurzfassung}
\include{abstract}

%%%----------------------------------------------------------
\mainmatter           %Hauptteil (ab hier arab. Seitenzahlen)
%%%----------------------------------------------------------

\include{einleitung}
\include{diplomschrift}
\include{latex}
\include{abbildungen}
\include{mathematik}
\include{literatur}
\include{drucken}
\include{word}
\include{schluss}

%%%----------------------------------------------------------
%%%Anhang
\appendix
\printglossaries
\include{anhang_a}	% Technische Ergänzungen
\include{anhang_b}	% Inhalt der CD-ROM/DVD
\include{anhang_c}	% Chronologische Liste der Änderungen
\include{anhang_d}	% Quelltext dieses Dokuments

%%%----------------------------------------------------------

%Literaturverzeichnis
\clearpage
\addcontentsline{toc}{chapter}{\bibname}

\bibliographystyle{bababbrv}
%options: babplain, babunsrt, bababbrv, babalpha
%         babplain-fl etc: first/lastname
%         babplain-lf etc: last/firstname
\bibliography{literatur}     %BibTeX-Datei literatur.bib


%%%----------------------------------------------------------

%%%Messbox zur Druckkontrolle
\include{messbox}

\end{document}
