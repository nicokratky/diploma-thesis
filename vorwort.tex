\chapter{Vorwort}

Dies ist \textbf{Version \htldiplDate} der \latex-Dokumentenvorlage für 
die Diplomarbeiten an der HTL Wiener Neustadt, basierend auf der
Vorlage für Abschlussarbeiten an der FH Hagenberg erstellt von Dr.\ Wilhelm
Burger\footnote{\url{http://www.fh-hagenberg.at/staff/burger/diplomarbeit/}}.

Die Verwendung dieser Vorlage ist jedermann freigestellt und an
keinerlei Erwähnung gebunden. Allerdings -- wer sie als Grundlage
seiner eigenen Arbeit verwenden möchte, sollte nicht einfach
("`ung'schaut"') darauf los werken, sondern zumindest die
wichtigsten Teile des Dokuments \emph{lesen} und nach Möglichkeit
auch beherzigen. Die Erfahrung zeigt, dass dies die Qualität der
Ergebnisse deutlich zu steigern vermag.

Der Quelltext zu diesem Dokument sowie das zugehörige
\latex-Paket sind in der jeweils aktuellen Version online
verfügbar unter
%
\begin{quote}
\url{http://unterricht.schermann.org/index.php/Diplomarbeitsvorlage_Latex}
\end{quote}
%
Trotz großer Mühe enthält dieses Dokument zweifellos Fehler und Unzulänglichkeiten
-- Kommentare, Verbesserungsvorschläge und passende Ergänzungen
sind daher stets willkommen, am einfachsten per E-Mail direkt an mich:
\begin{center}%
\begin{tabular}{l}
\nolinkurl{w.schermann@htlwrn.ac.at} \\
Wolfgang Schermann MSc \\
HTL Wiener Neustadt -- Informatik\\
Austria
\end{tabular}
\end{center}

\noindent
Übrigens, hier im Vorwort kann man kurz auf die Entstehung  des Dokuments eingehen.
Hier ist auch der Platz für allfällige Danksagungen (\zB an den Betreuer, 
den Begutachter, die Familie, den Hund, ...), Widmungen und philosophische 
Anmerkungen. Das sollte man allerdings auch nicht übertreiben und sich auf 
einen Umfang von maximal zwei Seiten beschränken.




