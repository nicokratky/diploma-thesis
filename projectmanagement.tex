\chapter{Project Management}
\label{ch:Project Management}

\section{Kanban}
\subsection{Description of Kanban}
Kanban is a technique to manage software development in a project. It utilizes the just-in-time (JIT) production system and easily reveals possible bottlenecks in the software development pipeline. Kanban provides information visual instead of textual, therefore the brain can extract information much faster. This has a reason, the human brain is able to processes visual information 60 000 times faster than text \cite{WhatIsKanban}. Kanban uses Sticky notes or some sort of cards on a board to display the information, by this method the pure text information is packed in a bigger picture that the brain can now process faster.
\subsection{How Kanban works}
Kanban has four core principles \cite{WhatIsKanban}.
\begin{enumerate}
    \item \textbf{Visualize Work}

    By visualizing the work and workflow in a project, it's easier to observe the flow of work through the whole Kanban system. Also problems like bottlenecks, blockers and queues can be spotted immediately.

    \item \textbf{Limit Work in Process}

    Kanban limits the amount of unfinished work in process, by doing so it reduces the time one item needs to travel through the Kanban system. That brings the advantage of less task switching and constantly reprioritizing.

    \item \textbf{Focus on Flow}

    By implementing work-in-progress (WIP) limits and team-driven policies the Kanban system can be optimized to improve the workflow. With tis methods the flow can be analyzed and it can indicate possible future problems.

    \item \textbf{Continuous Improvement}

    Once the Kanban system is established, teams can track their effectiveness by measuring flow, quality and throughput of work items they are assigned to. This can help a team a lot to improve and take their effectiveness to a higher level.
\end{enumerate}
\subsection{History of Kanban}

\section{Scrum}
\subsection{Description of Scrum}
\subsection{How Scrum works}
\subsection{History of Scrum}

\section{Scrumban}
\subsection{Description of Scrumban}
\subsection{How Scrumban works}
\subsection{History of Scrumban}
\subsection{Application of Scrumban in this Project}