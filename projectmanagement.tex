\chapter{Project Management}
\label{ch:Project Management}

\section{Kanban}

\subsection{Description of Kanban}
Kanban is a technique to manage software development in a project. It utilizes the just-in-time (JIT) production system and easily reveals possible bottlenecks in the software development pipeline. Kanban provides information visual instead of textual, therefore the brain can extract information much faster. This has a reason, the human brain is able to processes visual information 60 000 times faster than text \cite{WhatIsKanban}. Kanban uses Sticky notes or some sort of cards on a board to display the information, by this method the pure text information is packed in a bigger picture that the brain can now process faster.

\subsection{How Kanban works}
Kanban has four core principles \cite{WhatIsKanban}.
\begin{enumerate}
    \item \textbf{Visualize Work}

    By visualizing the work and workflow in a project, it's easier to observe the flow of work through the whole Kanban system. Also problems like bottlenecks, blockers and queues can be spotted immediately.

    \item \textbf{Limit Work in Process}

    Kanban limits the amount of unfinished work in process, by doing so it reduces the time one item needs to travel through the Kanban system. That brings the advantage of less task switching and constantly reprioritizing.

    \item \textbf{Focus on Flow}

    By implementing work-in-progress (WIP) limits and team-driven policies the Kanban system can be optimized to improve the workflow. With tis methods the flow can be analyzed and it can indicate possible future problems.

    \item \textbf{Continuous Improvement}

    Once the Kanban system is established, teams can track their effectiveness by measuring flow, quality and throughput of work items they are assigned to. This can help a team a lot to improve and take their effectiveness to a higher level.
\end{enumerate}

\subsection{History of Kanban}
Kanban was developed by Toyota in the late 1940s, the idea of Kanban came up from an unlikely source, the supermarket. They realized that the workers there ordered their goods if the stock was going near zero instead of order more goods if they are available at a vendor. So they ordered their goods after a ``just-in-time'' system. Kanban is Japanese for ``sign'' or ``signboard'', so Toyota invented a system that allowed teams to communicate better and showed them what work is needed to be done and when it's needed \cite{WhatIsKanban}.

\section{Scrum}

\subsection{Description of Scrum}
Scrum is a process framework to control and manage complex software development projects, it reduces the complexity of the whole system and focuses on a product that meets the business needs. ``Scrum makes clear the relative efficacy of your product management and development practices so that you can improve.'' \cite{ScrumGuide}

\subsection{How Scrum works}
Scrum consists of a number of entities that include:
\begin{itemize}
    \item \textbf{The Scrum Values} originally weren't included in the official Scrum guide, but due to most people considered them as a part of Scrum they were added in July 2016. These values are Commitment, Courage, Focus, Openness and Respect.

    \item \textbf{The Scrum Roles} are three roles every Scrum team needs to consist of. There is the ``Product Owner'', who needs to manage the Product Backlog. He needs to make sure the items are clearly expressed and in an order so that the team can achieve the best possible result. Theres also the ``Scrum Master'', who makes sure Scrum is used and understood properly. He also tells people outside of the Scrum team how they should interact with the team to maximize the output of the team. All of the other members in the Team are part of the ``Development Team''. Such a team is self-organized, cross-functional and there is no particular ranking in the team which means everybody is on the same level. The Development Team and only the Development Team is responsible to complete tasks from the Product Backlog at the end of each Sprint.

    \item \textbf{The Scrum Events} are used to create a regularity and minimize the effort of meetings outside the Scrum meetings. These events all have a fixed timespan and can't be shortened or lengthened. The Events embedded in Scrum are:
    \begin{enumerate}
        \item \textbf{Sprint}

        Each Sprint has a predefined timespan up to one month, in this period of time the Development Team should create a releasable increment to the product. An increment is the sum of all items in the Product Backlog that are finished within the Sprint. During the Sprint, nobody should make changes that would compromise the goal of the Sprint, also the stated quality of the product does not decrease. The scope of a Sprint can be clarified and re-negotiated with the Product Owner during the Sprint, when a Sprint finishes the next Sprint starts immediately.

        \item \textbf{Sprint Planning}

        The Sprint Planning Event takes place before each Sprint, in this meeting items from the Product Backlog are moved into the Sprint Backlog. These items doesn't have to be finished in one Sprint as a Sprint is defined by its time limit and not by the items in the Sprint Backlog.

        \item \textbf{Daily Scrum}

        In the Daily Scrum meeting the Development Team creates a plan for the next 24 hours. To do so the attendees analyze the work that has been done so far and with this knowledge they determine what to do until the next meeting. This meeting has a duration of at most 15 minutes and takes place every day in the same place at the same time.

        \item \textbf{Sprint Review}

        This meeting is held at the end of each Sprint, its purpose is to analyze the outcome and what items in the Product Backlog were actually finished in this Sprint. Attendees of this meeting are the Scrum team and the stakeholders, both groups should collaborate on the next things that could be done.

        \item \textbf{Sprint Retrospective}

        The main focus in this particular meeting lays on the improvement that can be done during the next Sprint. The topics discussed in the Sprint Retrospective are ``What went well in the Sprint'', ``What could be improved'' and ``What will we commit to improve in the next Sprint''.
    \end{enumerate}

    \item \textbf{Scrum Artifacts} represent the work that is to do is already is done to make it easier to inspect and adapt the particular work units. Therefore there are three Artifacts, the first one is the ``Product Backlog'', here defined are requirements that might be needed in the product. The Product Backlog evolves with the product itself, there will be requirements added and removed based on the results of the Sprints finished. The second Artifact is the ``Sprint Backlog'', in here are items that are assigned to do in the corresponding Sprint plus a plan how to accomplish the Sprint goal.
\end{itemize}
\subsection{History of Scrum}

\section{Scrumban}

\subsection{Description of Scrumban}

\subsection{How Scrumban works}

\subsection{History of Scrumban}

\subsection{Application of Scrumban in this Project}